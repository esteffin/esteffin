\documentclass[10pt]{llncs}

\usepackage{lipsum}
%\documentclass[10pt,a4paper]{article}
\usepackage[utf8]{inputenc}
\usepackage[english]{babel}
\usepackage{amsmath}
\usepackage{amsfonts}
\usepackage{amssymb}
\usepackage{comment}
%\usepackage{todos}

\author{Enrico Steffinlongo}
\institute{Universit\`{a} Ca' Foscari Venezia}
%\title{Relationship Based Access Control Model: Formal Definition and Verification}
\title{Enforcing security of browser extensions: a static analysis approach}

\begin{document}
\maketitle

\begin{abstract}
Lorem ipsum dolor sit amet, consectetur adipiscing elit. Curabitur convallis orci quis ipsum placerat, eu convallis velit cursus. Proin urna nunc, fringilla vel mi in, venenatis blandit sem. Fusce interdum at lectus non ultricies. Sed aliquet eleifend nibh, non viverra urna rhoncus ac. Fusce sit amet dictum eros. Ut nec sodales lorem. Phasellus egestas velit gravida orci mattis, in facilisis justo elementum. Donec massa tellus, fermentum ac est in, lacinia commodo sapien.
\end{abstract}

\section{Introduction}
In the last years web browsers evolved becoming to very powerful and complex software. To enhance their capabilities, modern browsers, provide very powerful extension architecture. Such architecture offers a simple API designed to build extensions that exposes to the extension the user interface, web communication and even internal components of the browser. While some extensions just modify the user interface, other performs to security critical operations like access to the cookie jar, download manager, password storage. Hence often the security of the user private data is strictly bound with the security of the extension installed in the browser. Since browser extensions interacts with web pages that are potentially malicious, they expand the attack surface of the browser and to mitigate this threat they rely on strong security principles like privilege separation \cite{ChromeExtSpec}.

Another problem is that browser extensions are written in JavaScript, and for JavaScript there are really few tools to help programmers in development. Moreover there are almost no tools to check statically web applications and browser extensions.

To achieve security warranties, various approaches have been adopted in different fields. Some researcher focus on definition of a robust architecture \cite{ChromeExtSpec} and \cite{PriviSep}, such as privilege separated ones where all components are separated in order to reduce the damage done by a compromised component. Other works instead focus on static analysis of JavaScript in order to give to programmers, information about the application. This second approach uses techniques derived from program analysis \cite{PrincipleProgramAnalysis} such flow logic \cite{FlowLogic}, typing \cite{TAJS}, and abstract interpretation \cite{LambdaJSMightVanHorn}.

Even if both approaches mitigate various threat, they are not a real panacea, because the trade-off 
The aim of this work is to combine both approaches 


\section{polok}
\lipsum[1-3]

\cite{*}


\bibliographystyle{splncs03}
\bibliography{thesis}

\end{document}