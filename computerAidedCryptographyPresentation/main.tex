\documentclass[11pt]{beamer}
\usetheme{Warsaw}
\usepackage[utf8]{inputenc}
\usepackage[english]{babel}
\usepackage{amsmath}
\usepackage{amsfonts}
\usepackage{amssymb}
\usepackage{stmaryrd}
\usepackage{infer}
\usepackage{listings}

\author{Enrico Steffinlongo}
\title{Simple Relational Correctness Proofs for
Static Analyses and Program Transformations}
\subtitle{By Nick Benton}
%\setbeamercovered{transparent} 
\setbeamertemplate{navigation symbols}{} 
%\logo{} 
\institute{Università Ca' Foscari - Computer science} 
\date{May 29, 2015}
%\subject{} 

\defbeamertemplate*{footline}{shadow theme}
{%
  \leavevmode%
  \hbox{\begin{beamercolorbox}[wd=.2\paperwidth,ht=2.5ex,dp=1.125ex,leftskip=.3cm plus1fil,rightskip=.3cm]{author in head/foot}%
    \hfill\insertshortauthor
  \end{beamercolorbox}%
  \begin{beamercolorbox}[wd=.8\paperwidth,ht=2.5ex,dp=1.125ex,leftskip=.3cm,rightskip=.3cm plus1fil]{title in head/foot}%
    \usebeamerfont{title in head/foot}\insertshorttitle\hfill\insertframenumber\,/\,\inserttotalframenumber%
  \end{beamercolorbox}}%
  \vskip0pt%
}

\lstset{language=C}
\lstset{commentstyle=\textit}
%\lstset{mathescape=true}
%\lstset{backgroundcolor=,rulecolor=}
\lstset{tabsize=2}
%\lstset{frame=lines}
\lstset{breaklines=true}
\lstset{basicstyle=\ttfamily}
\lstset{showstringspaces=false}

\newcommand{\Z}{\mathbb{Z}}
\newcommand{\B}{\mathbb{B}}
\newcommand{\St}{\mathbb{S}}
\newcommand{\bbracket}[1]{\llbracket #1 \rrbracket}
\newcommand{\cceil}[1]{\lceil #1 \rceil}
\newcommand{\while}[2]{\mathtt{while}~#1~\mathtt{do}~#2}
\newcommand{\ifc}[3]{\mathtt{if}~#1~\mathtt{then}~#2~\mathtt{else}#3}
\newcommand{\chevrons}[1]{\langle #1\rangle}

\begin{document}

\begin{frame}
\titlepage
\end{frame}

%\begin{frame}
%\tableofcontents
%\end{frame}

\begin{frame}{Soundness of Program optimization}
Lot of work on functional languages especially in
\begin{itemize}
\item formalization
\item validation
\end{itemize}
Few work on imperative programming languages
\begin{itemize}
\item seems trivial
\item ... but i's not
\end{itemize}
This work is to proposes three systems (one type based, and two Hoare-logics based) to prove correctness of optimization transformations.
\end{frame}

\begin{frame}[fragile]{Optimization transformations}
Transformation of a program to a semantically equivalent one in order to reduce the time used, or to decrease the resources used.\\
Typical imperative program optimization includes:
\begin{itemize}
\item constant propagation
\item dead-code elimination
\item program slicing
\item loop unrolling
\end{itemize}
\begin{tabular}{lclcl}
\begin{lstlisting}
X := 7;
Y := Y + 1;
X := 7;
Z := X;
\end{lstlisting}&$\Rightarrow$& 
\begin{lstlisting}
X := 7;
Y := Y + 1;
X := 7;
Z := 7;
\end{lstlisting}&$\Rightarrow$&
\begin{lstlisting}
X := 7;
Y := Y + 1;

Z := 7;
\end{lstlisting}
\end{tabular}
\end{frame}

\begin{frame}[fragile]{Optimization transformations: examples}
\begin{table}
\begin{tabular}{l}
\hline
\begin{lstlisting}
X := 3				
if X = 3 then		
    X := 7;		    
else               ==>     
    skip;                  X := 7;
Z := X + 1;                Z := 8;
\end{lstlisting}\\
\hline
\begin{lstlisting}
if X = 3 then            
    Y := X;                
else               ==>                
    Y := 3;                Y := 3
\end{lstlisting}\\
\hline
\begin{lstlisting}
X := -Y                    X := Y
Z := Z - X         ==>     Z := Z + X
X := -X                    
\end{lstlisting}\\
\hline
\end{tabular}
\caption{Transformation examples}
\label{Examples}
\end{table}
\end{frame}

\begin{frame}{The \texttt{while}-language: syntax}
In this work we will use the \texttt{while}-language.
$$
\begin{array}{rcll}
\mathbb{V} &=& \{X,Y,\dots\} &$ a set of variables$\\
n &\in& \Z &$ a number,$\\
b &\in& \B &$ a boolean literal$\\
iop &\in& \{+,-,\times,\dots\} \subseteq \Z \times \Z \rightarrow \Z &$ an integer operation$\\
bop &\in& \{<,=,\dots\} \subseteq \Z \times \Z \rightarrow \B &$ an integer to boolean$\\
&&&$operation$\\
lop &\in& \{\wedge,\vee,\dots \} \subseteq \B \times \B \rightarrow \B &$ a logical operation$\\
E &:=& n | X | E~iop~E &$ integer expressions$\\
B &:=& b | E~bop~E | \mathtt{not}~B | B~lop~B &$ boolean expressions$\\
C &:=& \mathtt{skip} | X:=E | C;C \\
&&| \mathtt{if}~B~\mathtt{then}~C~\mathtt{else}~C \\
&&| \mathtt{while}~B~\mathtt{do}~C &$ commands$\\
S&\in& \St=\mathbb{V}\rightarrow\Z&$A valid State$\\
\end{array}
$$
\end{frame}

\begin{frame}{\texttt{while}-Programs: semantics}
\begin{itemize}
\item Denotational semantics of Integer expression (similar to the one for Boolean expression)
$$
\begin{array}{rcl}
\bbracket{E} &\in& \St \rightarrow \bbracket{int}=\St \rightarrow \Z\\
\bbracket{n}S &=& n\\
\bbracket{X}S &=& S(X)\\
\bbracket{E_1~iop~E_2}S &=& (\bbracket{E_1}S)iop(\bbracket{E_2}S)\\
\bbracket{n}S &=& n\\
\end{array}
$$
\item Denotational semantics of commands 
$$
\begin{array}{rcl}
\bbracket{C}&\in& \St \rightarrow \St_\perp\\
\bbracket{\texttt{skip}} &=& \lambda S.\cceil{S}\\
\bbracket{X:=E} &=& \lambda S.\cceil{S[X \mapsto \bbracket{E}S]}\\
\bbracket{C_1;C_2} &=& \bbracket{C_2}^*\circ\bbracket{C_1}\\
\bbracket{\texttt{if} B \texttt{then} C_1 \texttt{else} C_2} &=& \lambda S.\bbracket{B}S \Rightarrow \bbracket{C_1}~|~\bbracket{C_2}\\
\bbracket{\texttt{while}B\texttt{do}C} &=& fix~f.\lambda S.\bbracket{B}S \Rightarrow f^*(\bbracket{C}S)|\cceil{S}\\
\end{array}
$$
\end{itemize}
\end{frame}

\begin{frame}{Dependency, Dead Code and Constant (DDCC)}
The DDCC type system is used to prove correctness of some program transformations.
\begin{itemize}
\item non-standard type system
\item it derives typed equality between expressions and commands
\item it works on pairs of programs
\item has simple types for expressions
\item has maps from variables to simple types for states
\item can be seen as a non-interference type system 
\item it captures only decisions based on known variables. So it is not able to capture patterns like in example 2 in table \{\ref{Examples}\}
\item does not capture code-motion transformation
\end{itemize}
Using DDCC we can prove the equivalence of the example 1 in table \{\ref{Examples}\}
\end{frame}

\begin{frame}{DDCC}
A simple type $\phi_\tau:=\mathbb{F}_\tau~|~\{c\}_\tau~|~\Delta_\tau~|~\mathbb{T}_\tau$ where $\tau\in\{\mathtt{int,bool}\}$ and $c$ is a constant.
\begin{itemize}
\item $\mathbb{F}_\tau$ is an empty type
\item $\{c\}_\tau$ is the type of a constant $c$ ($5~5 \in \{5,5\}_{int}$)
\item $\Delta_\tau$ is the type of an unknown expression (if we do not know the value of X $(X + X) \in \Delta_{int}$)
\item $\mathbb{T}_\tau$ is the type of an expression that we do not care any more.
\end{itemize}
A state type $\Phi := -~|~\Phi,X:\phi_{int}$ is a map from variable to simple types.\\
Judgements are of the form
\begin{itemize}
\item $\vdash E\sim E':\Phi\Rightarrow\phi_\tau$ for expressions,
\item $\vdash C\sim C':\Phi\Rightarrow\Phi'$ for commands.
\end{itemize}
We will use $\vdash C : \Phi \Rightarrow \Phi$ as shorthand for $\vdash C\sim C : \Phi \Rightarrow \Phi$
\end{frame}

\begin{frame}{DDCC core judgements}
Some simple DDCC core judgements for are:
\begin{itemize}
\item $\vdash n \sim n : \Phi \Rightarrow \{n\}_{int}$
\item $\vdash X \sim X : \Phi,X:\phi_{int} \Rightarrow \phi_{int}$
\item $\vdash \mathtt{skip} \sim \mathtt{skip} : \Phi \Rightarrow \Phi$
\item $\inferrule{\vdash C_1\sim C'_1:\Phi\Rightarrow\Phi'\\\vdash C_2\sim C'_2:\Phi'\Rightarrow\Phi''}{\vdash (C_1;C_2)\sim (C'_1;C'_2) : \Phi\Rightarrow \Phi''}$
\item $\inferrule{\vdash B \sim B':\Phi\Rightarrow\Delta_{\mathtt{bool}}\\ \vdash C\sim C' : \Phi\Rightarrow\Phi}{\vdash (\while{B}{C})\sim (\while{B'}{C'}) : \Phi\Rightarrow \Phi}$
\end{itemize}
These rules are able to prove relations on the same phrase to itself.
\end{frame}

\begin{frame}{DDCC judgements}
Judgements used to prove equivalences of transformed programs.\\
\vspace{.3cm}
\begin{small}
\begin{tabular}{lr}
sequential unit laws & \begin{tiny}$\inferrule {\vdash C\sim C:\Phi\Rightarrow\Phi}{\vdash (\mathtt{skip};C)\sim C:\Phi\Rightarrow\Phi'}$\end{tiny}\\
associativity & \begin{tiny}$\inferrule {\vdash (C_1;C_2);C_3:\Phi\Rightarrow\Phi'}{\vdash ((C_1;C_2);C_3)\sim(C_1;(C_2;C_3)):\Phi\Rightarrow\Phi'}$\end{tiny}\\
commuting conversion for conditionals\\
loop unrolling\\
self-assignment elimination & \begin{tiny}$\vdash (X:=X)\sim\mathtt{skip}:\Phi,X:\phi_{int}\Rightarrow\Phi,X:\phi_{int}$\end{tiny}\\
dead-assignment elimination & \begin{tiny}$\vdash (X:=E)\sim\mathtt{skip}:\Phi,X:\phi_{int}\Rightarrow\Phi,X:\mathbb{T}_{int}$\end{tiny}\\
equivalent branches for conditionals & \begin{tiny}$\inferrule{\vdash C_1\sim C_2:\Phi\Rightarrow \Phi'}{\ifc{B}{C_1}{C_2}\sim C_1:\Phi\Rightarrow \Phi'}$\end{tiny}\\
constant folding & \begin{tiny}$\inferrule {\vdash F_\tau:\Phi\Rightarrow \{c\}_\tau}{\vdash F_\tau\sim c:\Phi\Rightarrow \{c\}_\tau}$\end{tiny}\\
known branch for conditional & \begin{tiny}$\inferrule{\vdash B:\Phi\Rightarrow \{true\}\\ \vdash C_1\sim C' :\Phi\Rightarrow\Phi'}{\ifc{B}{C_1}{C_2}\sim C':\Phi\Rightarrow \Phi'}$\end{tiny}\\
dead while &\begin{tiny}$\inferrule{\vdash B:\Phi \Rightarrow \{false\}}{\vdash (\while{B}{C}\sim \mathtt{skip}:\Phi\Rightarrow\Phi}$\end{tiny}\\
divergence for while\\
\end{tabular}
\end{small}
\end{frame}

\begin{frame}{Relational Hoare Logic}
To increase the capabilities of the analysis this work proposes a system based on Relational Hoare Logics. It is based on\\
$$
\begin{array}{rcll}
GE &:=& n~|~X\chevrons{1}~|~X\chevrons{2}~|~GE~iop~GE:&$ generalized expressions$\\
\Phi &:=& b~|~GE~bop~GE~|~\mathtt{not}\Phi~|~\Phi~lop~\Phi:&$ relational assertions$\\
\end{array}
$$

Judgements are of the form:
$$
\vdash C\sim C':\Phi\Rightarrow\Phi'
$$
\end{frame}

\begin{frame}{Relational Hoare Logic: semantics}
The semantics of GE, and $\Phi$ are:\\
$$
\begin{array}{rcl}
\bbracket{GE}& \in & \St\times\St\rightarrow\Z\\
\bbracket{n}(S_1,S_2)& = & n\\
\bbracket{X\chevrons{1}}(S_1,S_2)& = & S_1(X)\\
\bbracket{X\chevrons{2}}(S_1,S_2)& = & S_2(X)\\
\bbracket{E~iop~F}(S_1,S_2)& = & (\bbracket{E}(S_1,S_2))~iop~(\bbracket{F}(S_1,S_2))\\
\\
\bbracket{\Phi}&\in& \St \times \St\\
&=&\{(S,S')|\chi_\Phi(S,S')=true\}\\
\chi_{true}(S,S')&=&true\\
\chi_{false}(S,S')&=&false\\
\chi_{E~bop~F}(S,S')&=&(\bbracket{E}(S_1,S_2))~bop~(\bbracket{F}(S_1,S_2))\\
\chi_{\Phi~lop~\Phi'}(S,S')&=&(\chi_{\Phi}(S_1,S_2))~lop~(\chi_{\Phi'}(S_1,S_2))\\
\chi_{\mathtt{not}\Phi}(S,S')&=&\neg(\chi_{\Phi}(S_1,S_2))\\
\end{array}
$$
\end{frame}

\newcommand{\vchev}[2]{#1\chevrons{#2}}
\begin{frame}{RHL: core}
Some simple RHL core judgements for are:
\begin{small}
\begin{itemize}
\item $\vdash \mathtt{skip} \sim \mathtt{skip} : \Phi \Rightarrow \Phi$
\item $\vdash X:=E\sim Y:=E':\Phi[\vchev{E}{1}/\vchev{X}{1},\vchev{E'}{2}/\vchev{Y}{2}]\Rightarrow\Phi$
\item $\inferrule{\vdash C_1\sim C'_1:\Phi\Rightarrow\Phi'\\\vdash C_2\sim C'_2:\Phi'\Rightarrow\Phi''}{\vdash (C_1;C_2)\sim (C'_1;C'_2) : \Phi\Rightarrow \Phi''}$
\item $\inferrule{\vdash C\sim C' : \Phi \wedge (\vchev{B}{1}\wedge\vchev{B'}{2}) \Rightarrow \Phi \wedge (\vchev{B}{1}=\vchev{B'}{2})}{\vdash (\while{B}{C})\sim (\while{B'}{C'}) :\\ \Phi \wedge (\vchev{B}{1}=\vchev{B'}{2}) \Rightarrow \Phi \wedge \mathtt{not} (\vchev{B}{1}\vee\vchev{B'}{2})} $
\end{itemize}
\end{small}
\end{frame}

\begin{frame}{RHL: judgements}
Judgements used to prove equivalences of transformed programs.\\
\vspace{.3cm}
\begin{small}
\begin{tabular}{lr}
\hline\\
falsity & \begin{tiny}$C\sim C':false\Rightarrow\Phi$\end{tiny}\\
\hline\\
dead-assignment elimination & \begin{tiny}$\vdash (X:=E)\sim\mathtt{skip}:\Phi[\vchev{E}{1}/\vchev{X}{1}]\Rightarrow\Phi$\end{tiny}\\
\hline\\
common branch & \begin{tiny}$
\inferrule
	{\vdash C\sim D :\Phi\wedge\vchev{B}{1}\Rightarrow\Phi'\\\\ \vdash C\sim D :\Phi\wedge\mathtt{not}\vchev{B}{1}\Rightarrow\Phi'}
	{\ifc{B}{C}{C'}\sim D:\Phi\Rightarrow \Phi'}$\end{tiny}\\
\hline\\
dead while &\begin{tiny}$\vdash (\while{B}{C}\sim \mathtt{skip}:\Phi\wedge\mathtt{not}\vchev{B}{1} \Rightarrow \Phi\wedge\mathtt{not}\vchev{B}{1}$\end{tiny}\\
\hline
\end{tabular}
\end{small}
\end{frame}


%\begin{frame}{assas}
%$$
%\inferrule*[width=10em,lab=(PE-Cond)]
%{ \sqsubseteq \rho}
%{\hat{v} \rho}
%$$
%\end{frame}

\end{document}